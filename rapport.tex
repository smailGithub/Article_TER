\documentclass{article}

\usepackage[francais]{babel}
\def\printlandscape{\special{landscape}}    % Works with dvips.
%\usepackage{pstricks,pst-node,pst-tree}
%\usepackage{amssymb}
\usepackage[utf8]{inputenc}
\usepackage[T1]{fontenc} 
\usepackage{fancybox} % for shadow and Bitemize
\usepackage{alltt}
\usepackage{graphicx}
%\usepackage{epsfig}
%\usepackage{fullpage}
%\usepackage{fancyhdr}
%\usepackage{moreverb}
%\usepackage{xspace}
\usepackage[colorlinks,hyperindex,bookmarks,linkcolor=blue,citecolor=blue,urlcolor=blue]{hyperref}

\usepackage{wrapfig}
\usepackage{epsf}

\title{Rapport de TER}
\author{Martin Strecker\\
\url{http://www.irit.fr/~Martin.Strecker}}
\date{\today}
         
\begin{document}

\maketitle
\tableofcontents

\begin{abstract}
Résumé du contenu du document.
\end{abstract}




%-----------------------------------------------------------
\section{Résumé}
\label{hints}

\section{Introduction}
\label{hints}

\section{Fondamenteux}
\label{hints}
\subsection{Langages et grammaires}
\subsubsection{Défintion}
		Un langage formel est un ensemble de mots constitués de symboles qui appartiennent à son alphabet.
Un langage formel est décrit par une grammaire. 

De manière générale, une grammaire est définie par un quadruplet:
\begin{itemize}
\item $N$ : l’ensemble des non-terminaux utilisés pour décrire les règles de productions
\item $X$ : l’ensemble des terminaux, c’est à dire les symboles ou encore l’alphabet
\item $P$ : l’ensemble des règles de production
\item $S$ : l’axiome, c’est un élément de N
\end{itemize}

Ainsi, la notation: $G(L) = <N, X, P, S>$ décrit la grammaire G associée au langage L.

Les grammaires sont analysées par des automates. Il existe plusieurs types d’analyses de grammaires qui font appels à plusieurs types d’automates.
Dans le monde de la compilation l’analyseur syntaxique fait référence à l’algorithme qui met en oeuvre l’automate d’analyse d’une grammaire.
Dans cet écrit, nous nous attarderons sur deux types d’analyseurs: les analyseurs de type LL et ceux de type L(AL)R.
\subsubsection{Grammaire ambigue}
	On dit qu’une grammaire est ambiguë  lorsqu’on peut trouver deux arbres de dérivation différents pour le même mot.
Les grammaires ambiguë pose un problème lors de la compilation, c’est pour ça qu’il est préférable de les transformer en grammaire non ambiguë si c’est possible.

\subsection{Analyse Lexicale}
\subsubsection{Défintion}
	l’analyse lexicale consiste à découper une chaîne de caractère en unités lexicales ou lexèmes à la demande de l’analyseur lexicale.
Il est définit par un ensembles d’expressions relationnelle qui exigent certaines séquences de caractère pour former les lexèmes.
\subsubsection{Segmentation}
	La segmentation est le fait de séparer les différentes sections d’une chaînes de caractères, par exemple pour une phrase l’ordinateur la considère comme une chaîne de caractère et non pas une suite de mot, le rôle de la segmentation est donc de faire une séparation entre ces mots selon le caractère de séparation dans ce cas le caractère espace. 
\subsubsection{Unités lexicales}
	Une unité lexicale ou un lexème est une chaîne de caractère qui correspond à un symbole. à l’aide du processus de segmentation, on peut extraire à partir d’un flux de caractères entrant une suite d’unités lexicales, ensuite c’est l’analyseur lexicale qui traitent ces lexème et les rangent dans des catégories d’entités lexicales.

\subsection{Analyse syntaxique}
\subsubsection{Défintion}
	Le rôle de l’analyse syntaxique est de savoir si une phrase appartient à la syntaxe d’un langage.
A partir du flot de lexèmes construits par l’analyse lexicale dans un premier temps, l’analyse syntaxique permet de générer un arbre de syntaxe abstraite.
Cet arbre est construit à base d’un ensembles de règles définissant une grammaire formelle sur laquelle est basé le langage en question.
l’analyse syntaxique permet plus particulièrement de détecter les erreurs de syntaxe en continuant tout de même l’analyse pour éviter les cycles de compilation/correction pour les développeurs.
un analyseur syntaxique doit retracer le cheminement d’application des règles qui ont menées à l’axiome. Pour ça, il existe deux types d’analyse:
\subsubsection{Analyse descendante}
	Le principe de l’analyse ascendante  est de partir de l’axiome en suivant les règles de production afin de retrouver le texte analysé. Ce type d’analyse procède en découpant le texte petit à petit jusqu'à retrouver les unité lexicale. L’analyse LL est un exemple d’analyse descendante.
\subsubsection{Analyse ascendante}
	L’analyse descendante d’une autre part, procède contrairement à l’analyse syntaxique en retrouvant le cheminement à partir du texte analysé. Ce type d’analyse essaye de regrouper les unité lexicale entre elles pour retrouver l’axiome. L’analyse LR est un exemple d’analyse ascendante.

\subsection{LL *}
\subsubsection{Défintion}
	Descente récursive ou prédicative: pour les grammaires simple ou le premier symbole terminal fournie des informations suffisantes pour choisir la règle de production.
Pas possible si grammaire récursive à gauche.
Besoin de factorisation à gauche lorsque 2 règles commencent par le même lexème.
Mais cette méthode à une faiblesse, c’est qu’elle doit toujours prédire qu’elle règle utiliser.

\subsection{LL A R}
\subsubsection{LR}
	Dérivation à droite permet de rapporter le choix de la règle de production à utiliser.
Elle commence du bas vers le haut.
L’algorithme s’arrête quand tous les caractères ont été lus. La chaîne est accépté si la partie analysée se réduit à l’axiome. 
\subsubsection{LALR}
	Utilisé lorsque le traitement des données doit répondre à de multiples cas et lorsque la
résolution par la programmation “standard” ne permettait pas une maintenance facile.
Yacc et GNU Bison sont des analyseur grammaticaux.

\section{Comparaison entre les outils}
\label{hints}
\subsection{YACC/BISON/CUP/JFLEX}
\subsubsection{YACC}
	Utilisé lorsque le traitement des données doit répondre à de multiples cas et lorsque la
résolution par la programmation “standard” ne permettait pas une maintenance facile.
Yacc et GNU Bison sont des analyseur grammaticaux.
\subsubsection{Spécifications}
	La spécification est l’ensemble des données qui permettront à YACC de générer l’analyseur. Elle décrit le langage qui sera reconnu sous forme de règles de grammaires. De cette façon l’analyseur a les connaissances pour définir si un flux donné en entrée est syntaxiquement correcte par rapport à sa spécification.
Cependant, un analyseur syntaxique est souvent utilisé dans le contexte de traduction de langage. La spécification permet cette fonctionnalité puisqu’il est possible de spécifier des actions associées aux règles de grammaire.

\subsection{ANTLR}
\subsubsection{Défintions}
	ANTLR est un Générateur de  Parseur public, il permet d’utiliser plusieurs langages et intégre la description Lexical/Syntaxique.
Grammaire LL k
ANTLR est un PARSEUR qui permet d’effectuer toutes les tâches où les autres PARSEUR échoue, il permet aussi d’établir le lien entre le SCANNER, rapporter les erreurs et aussi construire l’arbre syntaxique, ceci sont des actions généralement fait à la main dans le cas d’autres PARSEUR.
\subsubsection{Features}
	Différencier entre l’analyse Syntaxique et Lexical
	Faciliter de construction d’un arbre syntaxique
	Faciliter de détection d’erreur
	Portabilité en C ….
	…........................
\subsubsection{Elèments de langages}
	Grâce à des éléments de langages, ANTRLR différencie entre  le Prédicat Syntaxique et Semantique  comme l’exemple précèdent.
\subsubsection{Grammaire}
	ANTLR à aussi besoin de beacoup mois de LL pour différencier entre les alternatives de fins
exemple : LL3 pour un autre Parseur Vs LL 1 pour ANTLR
ANTLR utilise des labels pour accéder au attributs , on utilisant ce procédé, la grammaire et beacoup plus lisible.
PARSEUR classique  ce référence à la table des symboles pour résoudre une ambiguïté  syntaxique
ANTLR  Utilise un prédicat syntaxique grâce à ces éléments de langages.
Une description ANTLR contient à la fois la spécification ( Analyseur Lexical) et la grammaire ce qui permet de les grouper dans un seule fichier. Ceci est fait grâce au élément ANTLR (et les labels) qui permet de détecter et de différencier entre les grammaires et les TOKENS
ANTLR permet aussi d’utiliser plusieurs Analyseur lexical dans la même description ANTLR
\subsubsection{Gestion d'erreurs}
	Heuristique bien définie et efficace (gestion simple)
	PARSER EXCEPTION HANDLING pour une gestion plus sophistiquer
\subsubsection{L'arbre}
	Grâce à ces éléments de langage, ANTLR réussi à construire automatiquement son AST ABSTRACT SYNTAXIC TREE
ANTLR génère du code C et Cpp pour un Parseur de descente récursive ou toutes les 
règles de grammaires sont réalisé par des fonctions C / C++

\subsection{XTEXT}
\subsubsection{Défintions}
	Xtext est un framework pour le développement de langages de programmation et de DSL Domain Specific programming language .
DSL est un langage de programmation dont les spécifications sont à un domaine d’applications précis , la construction des langages dédiés diffère fondamentalement de celle d’un langage classique, le processus de développement  peut s'avérer très complexe, sa conception nécessite une double compétence sur le domaine à traiter et en développement informatique(exp :SQL:destiné à interroger ou manipuler une base de données relationnelle)
Le framework Xtext s’appuie sur sur une grammaire générée ANTLR ainsi que le framework de modelisation EMF.
Xtext couvre tous les aspects d’un IDE moderne : parseur, compilateur ,interpreteur et integration complete dans l’environnement de developpement Eclipse.
Xtext fournit un environnement convivial aux utilisateurs d’un DSL.
Parmi les fonctionnalités qu’offre Xtext:
*coloration syntaxique; suivant les éléments de la grammaire , Xtext propose une *coloration syntaxique entièrement personnalisable.
*auto complétion: auto complétion sur les éléments du langage 
*validation: Xtext valide le contenu de l'éditeur à la volée, produisant ainsi un retour direct à l’utilisateur en cas d’erreur de syntaxe.
*intégration avec d’autres composants Eclipse 

\section{Conclusion}
\label{hints}

\section{Références}
\label{hints}
\section{Some small hints}
\label{hints}
\subsection{German Umlauts and other Language Specific Characters}
\label{umlauts}
You can type german umlauts like 'ä', 'ö', or 'ü' directly in this file.
This is also true for other language specific characters like 'é', 'è' etc.
There are problems with automatic hyphenation when using language
specific characters and OT1-encoded fonts. In this case, use a
T1-encoded Type1-font like the Latin Modern font family (\verb#\usepackage{lmodern}#).
\subsection{References}
\label{references}
Using the commands \verb#\label{name}# and \verb#\ref{name}# you are able
to use references in your document. Advantage: You do not need to think
about numerations, because \LaTeX\ is doing that for you.
For example, in section \ref{dividing} on page \pageref{dividing} hints for
dividing large documents are given.
Certainly, references do also work for tables, figures, formulas\ldots
Please notice, that \LaTeX\ usually needs more than one run (mostly 2) to
resolve those references correctly.
\subsection{Dividing Large Documents}
\label{dividing}
You can divide your \LaTeX-Document into an arbitrary number of \TeX-Files
to avoid too big and therefore unhandy files (e.g. one file for every chapter).
For this, you insert in your main file (this one) for every subfile
the command '\verb#\input{subfile}#'. This leads to the same behavior
as if the content of the subfile would be at the place of the
\verb#\input#-Command.


\appendix
%-----------------------------------------------------------
\section{Introduction}\label{sec:intro}

Prérequis:
\begin{itemize}
\item Un fichier \texttt{*.tex}, par exemple \texttt{rapport.tex}, qui
  contient le texte.
\item Un fichier \texttt{*.bib}, par exemple \texttt{rapport.bib}, qui
  contient les références bibliographiques.
\end{itemize}

Utilisation de Latex:
\begin{itemize}
\item Lancer \texttt{pdflatex rapport} pour compilation du fichier
  \texttt{rapport.tex}
\item Lancer \texttt{biblatex rapport} pour compilation du fichier
  \texttt{rapport.bib}
\item Répéter \texttt{pdflatex rapport} deux fois pour prendre en
  compte des modifications
\end{itemize}

Documentation: 
\begin{itemize}
\item \LaTeX: \url{http://www.latex-project.org/} 
\item \TeX users group: \url{http://www.tug.org/}
\end{itemize}


%-----------------------------------------------------------
\section{Présentation du projet}

\subsection{But du projet}
On a utilisé
\cite{lindholm99_java_virtual_machin_specif,moore89_system_verif} et
aussi \cite{strecker02_verif_java_compil}.

\subsection{Implantation}

\begin{figure}[htbp]
  \centering
%  \includegraphics[scale=0.5]{foo}
  \caption{Une petite image}
  \label{fig:im}
\end{figure}

Inclusion d'images en format PDF (création par exemple avec
XFig\footnote{Page web: \url{http://www.xfig.org/}})

\subsection{Spécialités}

Facile à utiliser: Listes d'éléments numérotés:
\begin{enumerate}
\item premier élément
\item deuxième élément
\item aussi imbriqués:
%
\begin{itemize}
\item Ici, une liste non numérotée
\item avec un autre élément
\end{itemize}
%
\end{enumerate}

Mathématiques:
\begin{itemize}
\item Sous- et super-scripts: $a_n$ et $b^k$, utiliser des accolades
  pour des expressions plus complexes: $x^{(y^z)}$

\item Symboles mathématiques, tels que $\sum_{i=0}^{n} f(i)$

\item Caractères de l'alphabet grecque: $\Gamma$ ou calligraphiques: ${\cal D}$
\end{itemize}

Fontes de caractères spécifiques:
\begin{itemize}
\item \emph{Italique}
\item \textbf{Gras}
\item \texttt{typewriter}
\item ou combiné \emph{\textbf{italique-gras}}
\end{itemize}

%-----------------------------------------------------------
\section{Gestion du projet}

\begin{tabular}[h]{|l|l|}
\hline
Dates              & Tâches  \\
\hline
\hline
Du 7 mars au 2 mai & 1~ière itération \\
                   & 2~ième  itération \\
\hline
\end{tabular}

Je me réfère à la section~\ref{sec:intro} et la figure~\ref{fig:im}.

%-----------------------------------------------------------
\bibliography{rapport}
\bibliographystyle{abbrv}
\end{document}

%%% Local Variables:
%%% mode: latex
%%% TeX-master: t
%%% coding: utf-8
%%% End:
